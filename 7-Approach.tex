\Chapter{RESEARCH APPROACH AND THESIS ORGANIZATION}
\label{sec:approach}

This chapter will present the research approach used to meet the objectives detailed in \ref{sec:objectifs}. The link between the two works DORA-Explorer and RASS will follow and finally the document structure will be presented.

\section{Research approach}
Bringing risk-awareness and increasing robustness of swarm algorithms has been the primary focus of the master's work. Because the laboratory in which the master's degree was carried, the MIST Lab, focuses on space technologies, the approach chosen to meet the identified objectives was to develop meaningful algorithms for the exploration of space. In this regard, an exploration algorithm, DORA-Explorer, was built and enables efficient coverage of an unknown environment while avoiding its hazardous locations. Then, with a satisfactory exploration algorithm, focus switched towards building an efficient storing and routing algorithm: RASS. This algorithm is meant to be used by a robot swarm carrying the coverage task and collecting information in the process. The two algorithms are linked by their contribution to space exploration in the presence of risk. They are meant to be used side by side when exploring and gathering information about a new environment and should provide increased robustness to the swarm carrying the mission.

\section{Document structure}
The document's structure follows the one prescribed for a thesis with articles. Because I am first author only for the article on DORA-Explorer, this one will be directly included in the body of the thesis under its original article format whereas the second work, RASS, will be summarized. At the end of the thesis, appendices in relation with DORA-Explorer are presented. In appendices \ref{annexe: execution_loop}, \ref{annexe: intuition}, \ref{annexe: argos}, \ref{annexe: results}, \ref{annexe: physical} will be respectively presented the execution loop of DORA-Explorer, the intuition of the algorithm, the ARGoS environment from the simulations, some additional results and finally, a picture of the physical experiments carried in the lab. The document is structured as follows:

\begin{itemize}
    \item Chapter 1 gives an introduction on the research subject and presents the basic concepts upon which the work done in the course of the master's degree has been built. 
    \item Chapter 2 provides relevant contributions to the master's thesis found in the literature.
    \item Chapter 3 presents the common thread between the two research projects carried in the course of the master's degree
    \item Chapter 4 presents a fully decentralized and risk-aware exploration algorithm called DORA-Explorer. This work was published in the IEEE International Conference on Robotics and Automation (ICRA) in May 2022.
    \item Chapter 5 presents a fully decentralized and risk-aware routing algorithm called RASS. This work was published at the International Conference on Autonomous Agents and Multiagent Systems (AAMAS) as an extended abstract in May 2022 and was presented in the Autonomous Robots and Multirobot Systems (ARMS) workshop. 
    \item Chapter 6 discusses about the results obtained by the two algorithms and to what extent they satisfied the objectives of the master's degree. 
    \item Chapter 7 provides a summary of the works as well as some limitations of the algorithms presented and some interesting future research directions.
\end{itemize}

