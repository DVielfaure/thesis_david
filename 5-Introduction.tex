% Dans l'introduction, on présente le problème étudié et les buts
% poursuivis. L'introduction permet de faire connaître le cadre de la
% recherche et d'en préciser le domaine d'application. Elle fournit
% les précisions nécessaires en ce qui concerne le contexte de
% réalisation de la recherche, l'approche envisagée, l'évolution de
% la réalisation. En fait, l'introduction présente au lecteur ce
% qu'il doit savoir pour comprendre la recherche et en connaître la
% portée.

\Chapter{INTRODUCTION}\label{sec:Introduction}  % 10-12 lignes pour introduire le sujet.

The growing accessibility to robots has enabled the rise of a new field of robotic called swarm robotics. Such multi-robot systems are characterized by their ability to adapt to changing quantities of robots, the incompetence of the entities when considered individually and by the decentralized environment in which they evolve \cite{sahin2005swarm}. This new field of robotics has been widely inspired by insects where impressive results are achieved by combining efforts through self organization for the pursue of a common goal. Robot swarms are particularly appealing for exploration missions of large and unknown environments where a team effort can yield better coverage results.

Swarm systems should in theory be less prone to failures because of their inherent redundancy. However, it has been demonstrated that in practice, they do not benefit of this property and are, in some cases, even less reliable than traditional centralized robots systems in the presence of risk. It has been shown that individual failures often lead to the complete system failure, challenging the assumption that swarm systems are robust by nature \cite{bjerknes2013fault}. Designing specific tools for risk management in swarm systems is therefore of high importance \cite{prorok2021beyond}. The two algorithms DORA-Explorer and RASS that will be presented in the thesis are a step in this direction and aim at improving tolerance to risk in swarm exploration strategies.

\section{Definitions and Basic Concepts}  % environ 2-3 pages
The basic concepts upon which the two algorithms DORA-Explorer and RASS are built will be presented in the following section. 

\subsection{Swarm intelligence}
Swarm intelligence is a discipline that focuses on the mechanisms leading to global order in systems composed of many individuals coordinating between each other through local interactions only. Swarm intelligence can be seen in both natural and artificial systems where the collective behaviors are obtained in a fully decentralized fashion. Many examples of such systems can be found in nature, for example: colonies of ants, schools of fish, flocks of birds or herds of land animals \cite{Dorigo:2007}. Systems that fall into the discipline of swarm intelligence display the following characteristics: 

\begin{itemize}
    \item A swarm system comprises many individuals and should be able to adapt to varying quantities of individuals
    \item There is no central coordination, the behaviors of the agents emerge solely from local interactions
    \item The system is greater the the sum of its individuals. In other words, when taken individually the individuals are relatively incapable but collectively they are capable of achieving impressive results   
\end{itemize}

Such characteristics result in an adaptable system that should display a better tolerance to faults. Indeed, the absence of central coordination removes a bottleneck that could limit scalability and at the same time removes a central point of failure that could compromise the entire system.

\subsection{Swarm robotics}
\label{sec:designprinciples}
Swarm robotics falls into the artificial category of swarm intelligence and more specifically corresponds to the application of the discipline to the field of robotics. It is at the intersection of collective robotics and swarm intelligence. The design principles of swarm robotic systems are the following: 

\begin{itemize}
    \item Decentralized control
    \item Absence of leadership
    \item Absence of predefined roles
    \item Simple and local interactions
\end{itemize}

Self-organization is achieved from local interactions by the robot of the swarm. As a result, robots need to be aware of their neighbors and need to have the ability to communicate directly or indirectly through environmental stimuli. Of course, because of the engineering nature of swarm robotics, physical limitations arise. Bandwidth restrictions, communication capabilities, storage limitations and physical constraints all need to be taken into account when designing robot swarms.

\subsection{Virtual Stigmergy}
In nature, stirgmergy is a mechanism of indirect communication between agents of a system. It uses the environment for storing information that can be sensed and used by others. Stigmergy can be observed in social insects, for example ants that communicate between each other by laying down pheromones on the ground to guide others towards food \cite{bonabeau1999swarm}. Stigmergy is a fundamental aspect of swarm intelligence as it enables self-organization through local interactions.

The virtual stigmergy presented in \cite{pinciroliTuple2016} and implemented in the Buzz
programming language \cite{pinciroliBuzz2016} achieves consensus among
a group of robots using Conflict-free Replicated Data Types (CRDTs), represented as key-value pairs shared and replicated among the swarm members.  It is a valuable tool for easily programming swarm behaviors and is specifically designed to meet the requirements of such robotic systems. 

\subsection{Belief Maps}
Belief maps are commonly used in exploration algorithms. It renders a continuous surface into a discrete set of cells which is especially useful for engineering problems. Indeed, it lets you highlight specific regions of the environment and allows the designer of the algorithm to tune the level of precision at which the environment is represented. A finer granularity will give a more precise representation of the environment and as a results should offer more precision. However, finer granularity comes with higher computational costs. 

The cells of the belief maps are used to store information about the corresponding surface the cells represent. For example, it could store the absence of presence of a specific feature. In this case, because it is portraying an binary feature, the belief map is referred as an occupancy map. In belief maps, the value associated with the cells is a probability that indicates the confidence level that the cells contains a specific feature.

\subsection{Risk}
Recently, a lot of attention has been oriented towards risk-aware autonomous systems. Taking risk and uncertainty into account when designing robot systems is of capital importance as they are increasingly deployed on real-world scenarios and on safety-critical applications that include the transportation sector, aerospace systems or collaborative manufacturing. They evolve in uncontrolled environment and must face risks on a regular basis, motivating the need for risk-awareness. Additionally, excessive risk taking, or a complete absence of considering it, may not only jeopardize the robotic system but may also endanger components that are in its immediate environment which can sometime be human lives. For robot swarms tasked with the exploration of unknown environments, risk is generally location-based and takes the form of environmental hazards. 


\section{Problem elements}  % environ 3 pages
The exploration of hazardous environments using teams of robots comes with its own set of advantages and constraints.  Carrying the exploration mission with numerous robots should result in faster terrain coverage than in a single robot mission.  However, failures can considerably reduce the performance of the team and should therefore by avoided as much as possible. Not taking risk into account when exploring will inevitably lead to a decrease in performance of the team. This intuition is presented in Figure \ref{statement}.

\begin{figure*}[h]
    \centering
    \begin{subfigure}{0.63\textwidth}
        \includegraphics[width=\textwidth]{images/problemStatement1.png}
        \caption{}
        \label{statement1}
    \end{subfigure}
    \begin{subfigure}{0.63\textwidth}
        \includegraphics[width=\textwidth]{images/problemStatement2.png}
        \caption{}
        \label{statement2}
    \end{subfigure}
    \caption{Intuition of exploration of hazardous environments without risk-awareness.}
    \label{statement}
\end{figure*}

In figure \ref{statement1}, robots of the team start exploring the environment and gathering information about it. After a while, in figure \ref{statement2}, robots start experiencing failures due to environmental hazards. Here, the environmental hazards take the form of point radiation sources. Because of failures, the number of team members carrying the exploration mission decreases drastically. As a result, the exploration rate decreases and large portions of the environment remain uncovered. 

The intuition presented in figure \ref{statement} presents the problem statement of the exploration algorithm DORA-Explorer. Indeed, DORA-Explorer tries to solve this problem by introducing a conscience of risk in the motion control loop of the robots. Of course, because the exploration effort is carried by a swarm of robots, the algorithm needs to respect the design principles of swarm robotics.

Naturally, a similar problem statement can be expressed in terms of information gathering. If robots collecting information about the explored terrain are susceptible to data corruptions, not taking this risk into account will lead to poor data collection performance. This is the problem statement of RASS, the second algorithm of the thesis. Using risk-awareness, RASS tries to solve the problem by avoiding risky nodes of the system in its storage scheme.

\section{Research objectives} 
\label{sec:objectifs}
The main research objective is bringing risk-awareness to swarm algorithms used to perform exploration missions in hazardous environments. First, we focus on a risk-aware exploration algorithm, something that was lacking in the literature. Second, we focus on risk-awareness at the storing and routing level of the swarm infrastructure, also something that was missing from the literature. The research objectives are the following, the ones related to risk-aware exploration are first listed followed by the ones related to risk-aware storage and routing.


\subsection{Risk-aware exploration}
\begin{enumerate}
    \item Reduce failure rate compared to other state of the art algorithms;
    \item Achieve comparable terrain coverage compared to other state of the art algorithms;
    \item Respect swarm robotics design principles from section \ref{sec:designprinciples};
    \item Test real-world applicability with experiments on physical robots;
\end{enumerate}

\subsection{Risk-aware storage and routing}
\begin{enumerate}
    \item Reduce data corruption rate compared to other state of the art algorithms;
    \item Achieve comparable transfer speed compared to other state of the art algorithms;
    \item Respect swarm robotics design principles from section \ref{sec:designprinciples};
    \item Test real-world applicability with experiments on physical robots;
\end{enumerate}


\section{Thesis outline}  % 0.5 page
The remainder of thesis is structured as follows. In chapter \ref{sec:RevLitt}, a literature review of relevant related works is presented. The following topics will be studied in the literature review: swarm robotics, swarm programming, information sharing, routing mechanisms, swarm exploration strategies, risk in swarm robotics and some notable fault detection methods. Following the literature review, in chapter \ref{sec:approach}, the scientific approach used to meet the research objectives will be presented. Then, in chapter \ref{sec:Theme1} the algorithm DORA-Explorer which uses risk-awareness in its exploration strategy will be presented. Subsequently, chapter \ref{sec:Theme2} will cover RASS, a risk-aware storing and routing mechanism designed for robot swarms carrying exploration missions. A high level discussion on the results obtained throughout the master's degree in relation with the research objectives will follow in chapter \ref{sec:discussion}. Finally, in chapter \ref{sec:Conclusion}, a conclusion is presented alongside the limitations of the work and interesting future research directions.