% Résumé du mémoire.
%Le résumé est un bref exposé du sujet traité, des objectifs visés,
% des hypothèses émises, des méthodes expérimentales utilisées et de
% l'analyse des résultats obtenus. On y présente également les
% principales conclusions de la recherche ainsi que ses applications
% éventuelles. En général, un résumé ne dépasse pas quatre pages.

% Le résumé doit donner une idée exacte du contenu du mémoire ou de la thèse. Ce ne
% peut pas être une simple énumération des parties du document, car il
% doit faire ressortir l'originalité de la recherche, son aspect
% créatif et sa contribution au développement de la technologie ou à
% l'avancement des connaissances en génie et en sciences appliquées.
% Un résumé ne doit jamais comporter de références ou de figures.

\chapter*{RÉSUMÉ}\thispagestyle{headings}
\addcontentsline{toc}{compteur}{RÉSUMÉ}

L'exploration d'environnements inconnus est au coeur de plusieurs problèmes de robotique, parmi ceux-ci, des scénarios de sauvetage \cite{matos2016multiple} et des missions d'exploration extraplanétaire \cite{fong2005interaction}. Principalement, le problème d'exploration a été étudié pour des systèmes comportant un seul robot. Cependant, l'utilisation de plusieurs robots pour la réalisation de la mission pourrait s'avérer salutaire, en ce sens qu'avec une coordination adéquate, la vitesse à laquelle le terrain est parcouru devrait augmenter proportionnellement avec le nombre de robots dans le système \cite{burgard2005coordinated}. Ainsi, l'utilisation de systèmes multirobot, en opposition à un seul robot, se veut un domaine de recherche intéressant pour des applications d'exploration là où une importance est accordée à la vitesse à laquelle la mission s'effectue. Cependant, plusieurs défis demeurent, notamment en ce qui a trait aux fautes affectant les robots. Même si les systèmes multirobot présentent une certaine tolérance face aux risques en raison de leur intrinsèque redondance \cite{ramachandran2019resilience,wehbe2021probabilistic,winfield2006safety}, il a été démontré qu'en pratique, la robustesse qui les caractérise peut être moindre que celle d'un unique robot \cite{winfield2006safety}. En effet, une faute affectant un seul robot peut se propager à l'ensemble du système, causant une panne généralisée. En addition aux fautes, plusieurs autres problèmes affectent les systèmes multirobot. Parmi ceux-ci, on retrouve la coordination, la communication et le stockage de données. Le présent mémoire répondra à ces défis en présentant deux algorithmes de robotiques d'essaim:

\begin{itemize} 
\item DORA-Explorer: Distributed Online Risk-Aware Explorer
\item RASS: Risk-Aware Swarm Storage 
\end{itemize}

En ce qui concerne DORA-Explorer, sa contribution principale est d'introduire une conscience du risque dans un algorithme d'exploration pour essaim de robots. Disposer d'une stratégie d'exploration adaptée est particulièrement important puisque sans coordination, les robots du système couvriront les mêmes régions de l'environnement. Un tel comportement se traduirait que par de faibles gains d'information et n'est donc pas souhaitable d'un point de vue du système. Alors que cette coordination pourrait être orchestrée de façon optimale depuis une station centrale, il est en pratique impossible de le faire en raison de limitations au niveau de la connectivité des robots du système. En effet, une coordination centrale nécessiterait une haute bande passante et une connectivité continue. DORA-Explorer cherche à répondre à ce problème en proposant un algorithme d'exploration qui maximise la quantité d'information recueillie tout en minimisant la quantité de risque auquel les robots s'exposent. L'algorithme ne nécessite aucune coordination centrale et possède des couts de calculs très faibles ce qui le rend particulièrement adapté pour une utilisation sur essaims de robots. Le simulateur basé sur la physique ARGoS \cite{Pinciroli:SI2012} est utilisé par tester les performances de DORA-Explorer et finalement des expériences avec des robots physiques sont effectuées pour confirmer l'applicabilité de l’algorithme dans un scénario réel. Les résultats montrent que DORA-Explorer obtient des résultats d'exploration convaincants tout en réduisant considérablement la quantité de fautes qui affecte les robots du système lorsque comparé à d'autres solutions présentes dans la littérature.

En ce qui concerne RASS, sa contribution principale est d'apporter une conscience du risque dans un algorithme de stockage et de routage complètement décentralisé. Le stockage de données demeure un défi pour des systèmes multirobot en ce sens que la quantité d'information collectée ne fait que croitre avec le nombre de robots dans le système. Encore une fois, la connectivité hasardeuse qui caractérise ces systèmes \cite{amigoni2017multirobot} empêche l'envoi direct de l'information collectée vers une station de base pour un stockage permanent. Les robots doivent fréquemment stocker localement l'information collectée jusqu'à avoir un canal disponible pour l'envoyer vers un stockage permanent. De plus, puisque les robots ont généralement un rayon de communication limité, l'information doit être acheminée au travers de plusieurs autres robots avant de rejoindre la destination finale. Le système multirobot devient donc un système de stockage temporaire et décider où stocker l'information devient donc essentiel, particulièrement en présence de risque. Acheminer l'information par le chemin le plus court peut sembler naturel, cependant en présence de risque certains noeuds du système peuvent être trop dangereux pour être utilisés. Par exemple, envoyer de l'information à un robot situé à proximité d'une source de radiation pourrait causer des corruptions et n'est donc pas souhaitable. RASS cherche à répondre à ce problème en introduisant une conscience du risque pour déterminer où devraient être acheminées les données pour éviter le risque tout en percolant vers la station de base. Encore une fois, RASS est validée à l'aide du simulateur basé sur la physique ARGoS \cite{Pinciroli:SI2012} de même qu'à l'aide d'expériences sur des robots physiques. Nous obtenons des résultats convaincants montrant une diminution significative de la quantité d'information corrompue par les sources de radiations tout en conservant une vitesse de routage adéquate.
